% Options for packages loaded elsewhere
\PassOptionsToPackage{unicode}{hyperref}
\PassOptionsToPackage{hyphens}{url}
%
\documentclass[
]{article}
\usepackage{amsmath,amssymb}
\usepackage{lmodern}
\usepackage{iftex}
\ifPDFTeX
  \usepackage[T1]{fontenc}
  \usepackage[utf8]{inputenc}
  \usepackage{textcomp} % provide euro and other symbols
\else % if luatex or xetex
  \usepackage{unicode-math}
  \defaultfontfeatures{Scale=MatchLowercase}
  \defaultfontfeatures[\rmfamily]{Ligatures=TeX,Scale=1}
\fi
% Use upquote if available, for straight quotes in verbatim environments
\IfFileExists{upquote.sty}{\usepackage{upquote}}{}
\IfFileExists{microtype.sty}{% use microtype if available
  \usepackage[]{microtype}
  \UseMicrotypeSet[protrusion]{basicmath} % disable protrusion for tt fonts
}{}
\makeatletter
\@ifundefined{KOMAClassName}{% if non-KOMA class
  \IfFileExists{parskip.sty}{%
    \usepackage{parskip}
  }{% else
    \setlength{\parindent}{0pt}
    \setlength{\parskip}{6pt plus 2pt minus 1pt}}
}{% if KOMA class
  \KOMAoptions{parskip=half}}
\makeatother
\usepackage{xcolor}
\usepackage[margin=1in]{geometry}
\usepackage{color}
\usepackage{fancyvrb}
\newcommand{\VerbBar}{|}
\newcommand{\VERB}{\Verb[commandchars=\\\{\}]}
\DefineVerbatimEnvironment{Highlighting}{Verbatim}{commandchars=\\\{\}}
% Add ',fontsize=\small' for more characters per line
\usepackage{framed}
\definecolor{shadecolor}{RGB}{248,248,248}
\newenvironment{Shaded}{\begin{snugshade}}{\end{snugshade}}
\newcommand{\AlertTok}[1]{\textcolor[rgb]{0.94,0.16,0.16}{#1}}
\newcommand{\AnnotationTok}[1]{\textcolor[rgb]{0.56,0.35,0.01}{\textbf{\textit{#1}}}}
\newcommand{\AttributeTok}[1]{\textcolor[rgb]{0.77,0.63,0.00}{#1}}
\newcommand{\BaseNTok}[1]{\textcolor[rgb]{0.00,0.00,0.81}{#1}}
\newcommand{\BuiltInTok}[1]{#1}
\newcommand{\CharTok}[1]{\textcolor[rgb]{0.31,0.60,0.02}{#1}}
\newcommand{\CommentTok}[1]{\textcolor[rgb]{0.56,0.35,0.01}{\textit{#1}}}
\newcommand{\CommentVarTok}[1]{\textcolor[rgb]{0.56,0.35,0.01}{\textbf{\textit{#1}}}}
\newcommand{\ConstantTok}[1]{\textcolor[rgb]{0.00,0.00,0.00}{#1}}
\newcommand{\ControlFlowTok}[1]{\textcolor[rgb]{0.13,0.29,0.53}{\textbf{#1}}}
\newcommand{\DataTypeTok}[1]{\textcolor[rgb]{0.13,0.29,0.53}{#1}}
\newcommand{\DecValTok}[1]{\textcolor[rgb]{0.00,0.00,0.81}{#1}}
\newcommand{\DocumentationTok}[1]{\textcolor[rgb]{0.56,0.35,0.01}{\textbf{\textit{#1}}}}
\newcommand{\ErrorTok}[1]{\textcolor[rgb]{0.64,0.00,0.00}{\textbf{#1}}}
\newcommand{\ExtensionTok}[1]{#1}
\newcommand{\FloatTok}[1]{\textcolor[rgb]{0.00,0.00,0.81}{#1}}
\newcommand{\FunctionTok}[1]{\textcolor[rgb]{0.00,0.00,0.00}{#1}}
\newcommand{\ImportTok}[1]{#1}
\newcommand{\InformationTok}[1]{\textcolor[rgb]{0.56,0.35,0.01}{\textbf{\textit{#1}}}}
\newcommand{\KeywordTok}[1]{\textcolor[rgb]{0.13,0.29,0.53}{\textbf{#1}}}
\newcommand{\NormalTok}[1]{#1}
\newcommand{\OperatorTok}[1]{\textcolor[rgb]{0.81,0.36,0.00}{\textbf{#1}}}
\newcommand{\OtherTok}[1]{\textcolor[rgb]{0.56,0.35,0.01}{#1}}
\newcommand{\PreprocessorTok}[1]{\textcolor[rgb]{0.56,0.35,0.01}{\textit{#1}}}
\newcommand{\RegionMarkerTok}[1]{#1}
\newcommand{\SpecialCharTok}[1]{\textcolor[rgb]{0.00,0.00,0.00}{#1}}
\newcommand{\SpecialStringTok}[1]{\textcolor[rgb]{0.31,0.60,0.02}{#1}}
\newcommand{\StringTok}[1]{\textcolor[rgb]{0.31,0.60,0.02}{#1}}
\newcommand{\VariableTok}[1]{\textcolor[rgb]{0.00,0.00,0.00}{#1}}
\newcommand{\VerbatimStringTok}[1]{\textcolor[rgb]{0.31,0.60,0.02}{#1}}
\newcommand{\WarningTok}[1]{\textcolor[rgb]{0.56,0.35,0.01}{\textbf{\textit{#1}}}}
\usepackage{graphicx}
\makeatletter
\def\maxwidth{\ifdim\Gin@nat@width>\linewidth\linewidth\else\Gin@nat@width\fi}
\def\maxheight{\ifdim\Gin@nat@height>\textheight\textheight\else\Gin@nat@height\fi}
\makeatother
% Scale images if necessary, so that they will not overflow the page
% margins by default, and it is still possible to overwrite the defaults
% using explicit options in \includegraphics[width, height, ...]{}
\setkeys{Gin}{width=\maxwidth,height=\maxheight,keepaspectratio}
% Set default figure placement to htbp
\makeatletter
\def\fps@figure{htbp}
\makeatother
\setlength{\emergencystretch}{3em} % prevent overfull lines
\providecommand{\tightlist}{%
  \setlength{\itemsep}{0pt}\setlength{\parskip}{0pt}}
\setcounter{secnumdepth}{-\maxdimen} % remove section numbering
\ifLuaTeX
  \usepackage{selnolig}  % disable illegal ligatures
\fi
\IfFileExists{bookmark.sty}{\usepackage{bookmark}}{\usepackage{hyperref}}
\IfFileExists{xurl.sty}{\usepackage{xurl}}{} % add URL line breaks if available
\urlstyle{same} % disable monospaced font for URLs
\hypersetup{
  pdftitle={Homework 2},
  pdfauthor={TJ Sipin},
  hidelinks,
  pdfcreator={LaTeX via pandoc}}

\title{Homework 2}
\author{TJ Sipin}
\date{2022-07-06}

\begin{document}
\maketitle

\hypertarget{problem-1-statistical-thinking.}{%
\subsection{Problem 1 (Statistical
thinking).}\label{problem-1-statistical-thinking.}}

\begin{enumerate}
\def\labelenumi{(\alph{enumi})}
\tightlist
\item
  Consider
  \(\text{Cov}(\textbf{A}y) = \textbf{A}\text{Cov}(y)\textbf{A}^T,\)
  where \(\textbf{A} = (X^T X)^{-1} X^T\) and
  \(y = \beta X + \epsilon\). Taking the notation
  \[\hat \beta = (X^T X)^{-1} X^T y ,\] we have \[\begin{align}
  \text{Cov}(\hat \beta ) &= \text{Cov}((X^T X)^{-1} X^T y) \\
  &= \text{Cov}((X^T X)^{-1} X^T (X \beta + \epsilon)) \\
  &= \text{Cov}((X^T X)^{-1} X^T X \beta + (X^T X)^{-1} X^T \epsilon)\\
  &= \text{Cov} (I \beta + (X^T X)^{-1} X^T \epsilon)\\
  &= \text{Cov} (X^T X)^{-1} X^T \epsilon) \\
  &= \text{Cov}(\textbf{A}\epsilon) \\
  &= \textbf{A}\text{Cov}(\epsilon)\textbf{A}^T \\
  &= (X^T X)^{-1} X^T \sigma^2 ((X^T X)^{-1} X^T)^T \\
  &= \sigma^2 (X^T X)^{-1} X^T X \left ((X^T X)^{-1} \right )^T \\
  &= \sigma^2 (X^T X)^{-1}.
  \end{align}\]
\end{enumerate}

The variance of \(\hat\beta_1\) is given by the second diagonal of the
above matrix. That is,

\[
(i) \qquad 
\text{Var}(\hat \beta_1 ) = \left [ \sigma^2 (X^T X)^{-1}\right ]_{2,2}.
\]

\begin{enumerate}
\def\labelenumi{(\alph{enumi})}
\setcounter{enumi}{1}
\tightlist
\item
  Splitting the possible range of \(x\) into 100 mutually exclusive
  windows and collecting the equal amount of sample from each slot is
  more efficient in terms of estimating \(\beta_1\). So when we split
  the possible range of \(x\) into 100 \emph{mutually exclusive}
  windows, then we decrease variance of \(x\), which in turn, decreases
  the variance of \(\hat\beta\), as we see in \((i)\).
\end{enumerate}

\hypertarget{problem-2-prediction-and-confidence-intervals}{%
\subsection{Problem 2 (Prediction and confidence
intervals)}\label{problem-2-prediction-and-confidence-intervals}}

\begin{Shaded}
\begin{Highlighting}[]
\CommentTok{\# Regression for cars data}
\NormalTok{lm.out }\OtherTok{\textless{}{-}} \FunctionTok{lm}\NormalTok{(dist }\SpecialCharTok{\textasciitilde{}}\NormalTok{ speed, }\AttributeTok{data =}\NormalTok{ cars)}

\CommentTok{\# create new predictor }
\NormalTok{newx }\OtherTok{=} \FunctionTok{seq}\NormalTok{(}\FunctionTok{min}\NormalTok{(cars}\SpecialCharTok{$}\NormalTok{speed), }\FunctionTok{max}\NormalTok{(cars}\SpecialCharTok{$}\NormalTok{speed), }\AttributeTok{by =} \FloatTok{0.1}\NormalTok{)}

\CommentTok{\# get confidence interval for each element in newx}
\NormalTok{conf\_interval }\OtherTok{\textless{}{-}} \FunctionTok{predict}\NormalTok{(lm.out, }
                         \AttributeTok{newdata=}\FunctionTok{data.frame}\NormalTok{(}\AttributeTok{speed=}\NormalTok{newx),}
                         \AttributeTok{interval =} \StringTok{"confidence"}\NormalTok{,}
                         \AttributeTok{level =} \FloatTok{0.95}\NormalTok{)}

\CommentTok{\# get prediction interval for each element in newx}
\NormalTok{pred\_interval }\OtherTok{\textless{}{-}} \FunctionTok{predict}\NormalTok{(lm.out,}
                         \AttributeTok{newdata =} \FunctionTok{data.frame}\NormalTok{(}\AttributeTok{speed=}\NormalTok{newx),}
                         \AttributeTok{interval =} \StringTok{"prediction"}\NormalTok{,}
                         \AttributeTok{level =} \FloatTok{0.95}\NormalTok{)}
\CommentTok{\# plot data points}
\FunctionTok{plot}\NormalTok{(cars}\SpecialCharTok{$}\NormalTok{speed, cars}\SpecialCharTok{$}\NormalTok{dist, }
     \AttributeTok{main =} \StringTok{"Regression"}\NormalTok{,}
     \AttributeTok{xlab =} \StringTok{"Speed"}\NormalTok{,}
     \AttributeTok{ylab =} \StringTok{"Distance"}\NormalTok{)}

\CommentTok{\# regression line}
\FunctionTok{abline}\NormalTok{(lm.out, }\AttributeTok{col =} \StringTok{\textquotesingle{}lightblue\textquotesingle{}}\NormalTok{)}

\CommentTok{\# add lower end of CI}
\FunctionTok{lines}\NormalTok{(newx, conf\_interval[,}\DecValTok{2}\NormalTok{], }\AttributeTok{col =} \StringTok{\textquotesingle{}blue\textquotesingle{}}\NormalTok{, }\AttributeTok{lty =} \DecValTok{2}\NormalTok{)}
\CommentTok{\# add upper end of CI}
\FunctionTok{lines}\NormalTok{(newx, conf\_interval[,}\DecValTok{3}\NormalTok{], }\AttributeTok{col =} \StringTok{\textquotesingle{}blue\textquotesingle{}}\NormalTok{, }\AttributeTok{lty =} \DecValTok{2}\NormalTok{)}
\CommentTok{\# add lower end of PI}
\FunctionTok{lines}\NormalTok{(newx, pred\_interval[,}\DecValTok{2}\NormalTok{], }\AttributeTok{col =} \StringTok{\textquotesingle{}red\textquotesingle{}}\NormalTok{, }\AttributeTok{lty =} \DecValTok{2}\NormalTok{)}
\CommentTok{\# add upper end of PI}
\FunctionTok{lines}\NormalTok{(newx, pred\_interval[,}\DecValTok{3}\NormalTok{], }\AttributeTok{col =} \StringTok{\textquotesingle{}red\textquotesingle{}}\NormalTok{, }\AttributeTok{lty =} \DecValTok{2}\NormalTok{)}
\end{Highlighting}
\end{Shaded}

\includegraphics{hw2_files/figure-latex/problem 2-1.pdf}

The prediction interval is wider than the confidence interval since the
confidence interval pertains to the median (at a 95\% level), while the
prediction interval pertains to all points in the relationship of
distance and speed within a 95\% level.

\hypertarget{problem-3-weighted-least-squares.}{%
\subsection{Problem 3 (Weighted least
squares).}\label{problem-3-weighted-least-squares.}}

\begin{enumerate}
\def\labelenumi{(\alph{enumi})}
\tightlist
\item
  Derive the BLUE of \(\beta_0\) and \(\beta_1\) for
\end{enumerate}

\[
y_i \sim^{\text{indep}} N(\beta_0 + \beta_1 x_i, x_i^2 \sigma^2), \qquad i = 1, \dots, n.
\]

\emph{Solution.} Here, we have
\(\epsilon \sim N(0, \sigma^2 \textbf{V}),\) where
\(\textbf V = x_i^2I, \text{ for } i = 1,\dots, n\). The OLS estimator
is BLUE when \(\epsilon \sim (0, \sigma^2 I).\) Since \(\textbf V\) can
be decomposed as \(\textbf V = (x_i I) (x_i I)^T\), it follows that

\[
\text{Cov}((x_i I)^{-1} \epsilon) = \sigma^2 (x_iI)^{-1} x_i I (x_i I)^T ((x_i I )^{-1})^T = \sigma^2 I.
\]

This allows us to change the OLS problem to

\[
(x_i I)^{-1}Y = (x_i I)^{-1}X \beta + (x_i I)^{-1} \epsilon, \quad (x_i I)^{-1} \sim N(0, \sigma^2 I).
\]

Thus, we can get BLUE from the OLS estimator from the changed problem:

\[ (ii)
\begin{aligned}
\hat\beta &= (X^T((x_i I)^T)^{-1} (x_i I)^{-1} X)^{-1}((x_i I)^{-1} X)^T (x_i I)^{-1} Y \\
&= (X^T (x_i I x_i I^T)^{-1} X)^{-1} X^T (x_i I x_i I^T)^{-1} Y \\
&= (X^T (x_i I x_i I^T)^{-1} X)^{-1} X^T (x_i I x_i I^T)^{-1} Y \\
&= (X^T(x_i ^2 I)^{-1} X)^{-1} X^T(x_i^2 I)^{-1} Y,
\end{aligned}
\]

as long as \(X^T (x_i^2 I)^{-1}X\) is nonsingular. That is, if \(n > 1\)
and at least one \(x_i \neq 0\) . The BLUE of \(\beta_0\) is the first
row of the above matrix \((ii)\) and the BLUE of \(\beta_1\) is the
second row.

\begin{enumerate}
\def\labelenumi{(\alph{enumi})}
\setcounter{enumi}{1}
\tightlist
\item
\end{enumerate}

\begin{Shaded}
\begin{Highlighting}[]
\CommentTok{\# library(matlib)}
\FunctionTok{set.seed}\NormalTok{(}\DecValTok{2022}\NormalTok{)}
\NormalTok{x }\OtherTok{=} \FunctionTok{rnorm}\NormalTok{(}\DecValTok{20}\NormalTok{, }\DecValTok{30}\NormalTok{, }\DecValTok{5}\NormalTok{)}
\NormalTok{y }\OtherTok{=} \FunctionTok{rnorm}\NormalTok{(}\DecValTok{20}\NormalTok{, }\DecValTok{40}\NormalTok{, }\DecValTok{5}\NormalTok{)}


\NormalTok{lm.out}\FloatTok{.3}\NormalTok{b }\OtherTok{=} \FunctionTok{lm}\NormalTok{(y }\SpecialCharTok{\textasciitilde{}}\NormalTok{ x)}
\NormalTok{xm }\OtherTok{=} \FunctionTok{matrix}\NormalTok{(}\FunctionTok{c}\NormalTok{(}\FunctionTok{rep}\NormalTok{(}\DecValTok{1}\NormalTok{, }\AttributeTok{times =} \DecValTok{20}\NormalTok{), x), }\AttributeTok{nrow =} \DecValTok{20}\NormalTok{)}
\NormalTok{ym }\OtherTok{=} \FunctionTok{as.matrix}\NormalTok{(y)}
\NormalTok{q  }\OtherTok{=} \FunctionTok{diag}\NormalTok{(}\AttributeTok{x =}\NormalTok{ x}\SpecialCharTok{\^{}}\DecValTok{2}\NormalTok{, }\AttributeTok{nrow =} \DecValTok{20}\NormalTok{, }\AttributeTok{ncol =} \DecValTok{20}\NormalTok{)}

\NormalTok{beta.hat }\OtherTok{=} \FunctionTok{solve}\NormalTok{(}\FunctionTok{t}\NormalTok{(xm)}\SpecialCharTok{\%*\%}\FunctionTok{solve}\NormalTok{(q)}\SpecialCharTok{\%*\%}\NormalTok{xm)}\SpecialCharTok{\%*\%}\FunctionTok{t}\NormalTok{(xm)}\SpecialCharTok{\%*\%}\FunctionTok{solve}\NormalTok{(q)}\SpecialCharTok{\%*\%}\NormalTok{ym}
\NormalTok{beta.hat}
\end{Highlighting}
\end{Shaded}

\begin{verbatim}
##            [,1]
## [1,] 35.8811172
## [2,]  0.1709089
\end{verbatim}

The BLUE of \(\beta_0 = 38.28292797\) and the BLUE of
\(\beta_1 = 0.08383732\) based on what was derived in (a).

\begin{enumerate}
\def\labelenumi{(\alph{enumi})}
\setcounter{enumi}{2}
\tightlist
\item
\end{enumerate}

\begin{Shaded}
\begin{Highlighting}[]
\NormalTok{lm.out}\FloatTok{.3}\NormalTok{c }\OtherTok{=} \FunctionTok{lm}\NormalTok{(y }\SpecialCharTok{\textasciitilde{}}\NormalTok{ x, }\AttributeTok{weights =} \DecValTok{1}\SpecialCharTok{/}\NormalTok{(x}\SpecialCharTok{**}\DecValTok{2}\NormalTok{))}
\NormalTok{lm.out}\FloatTok{.3}\NormalTok{c}
\end{Highlighting}
\end{Shaded}

\begin{verbatim}
## 
## Call:
## lm(formula = y ~ x, weights = 1/(x^2))
## 
## Coefficients:
## (Intercept)            x  
##     35.8811       0.1709
\end{verbatim}

The results are the same as the results in (b). Here, the
\texttt{weights\ =} option is giving us the inverse of \(V^2\). This is
since
\(Y \sim N(\mu, V \sigma^2) \implies \frac{Y}{V} \sim N(\mu, \sigma^2)\).

\hypertarget{problem-4-wheezing-data.}{%
\subsection{Problem 4 (Wheezing data).}\label{problem-4-wheezing-data.}}

\begin{Shaded}
\begin{Highlighting}[]
\NormalTok{wheeze }\OtherTok{\textless{}{-}} \FunctionTok{read.csv}\NormalTok{(}\StringTok{\textquotesingle{}data/wheeze\_wmiss.csv\textquotesingle{}}\NormalTok{)}
\NormalTok{wheeze }\SpecialCharTok{\%\textgreater{}\%} \FunctionTok{head}\NormalTok{()}
\end{Highlighting}
\end{Shaded}

\begin{verbatim}
##   case t wheeze kingston age smoke
## 1    1 1      1        0   9     0
## 2    1 2      1        0  10     0
## 3    1 3      1        0  11     0
## 4    1 4      0        0  12     0
## 5    2 1      1        1   9     1
## 6    2 2      1        1  10     2
\end{verbatim}

\begin{Shaded}
\begin{Highlighting}[]
\FunctionTok{summary}\NormalTok{(wheeze)}
\end{Highlighting}
\end{Shaded}

\begin{verbatim}
##       case             t            wheeze          kingston     
##  Min.   : 1.00   Min.   :1.00   Min.   :0.0000   Min.   :0.0000  
##  1st Qu.: 4.75   1st Qu.:1.75   1st Qu.:0.0000   1st Qu.:0.0000  
##  Median : 8.50   Median :2.50   Median :0.0000   Median :0.0000  
##  Mean   : 8.50   Mean   :2.50   Mean   :0.2969   Mean   :0.4531  
##  3rd Qu.:12.25   3rd Qu.:3.25   3rd Qu.:1.0000   3rd Qu.:1.0000  
##  Max.   :16.00   Max.   :4.00   Max.   :1.0000   Max.   :1.0000  
##                                                                  
##       age            smoke     
##  Min.   : 9.00   Min.   :0.00  
##  1st Qu.: 9.75   1st Qu.:0.00  
##  Median :10.50   Median :1.00  
##  Mean   :10.50   Mean   :0.78  
##  3rd Qu.:11.25   3rd Qu.:1.00  
##  Max.   :12.00   Max.   :2.00  
##                  NA's   :14
\end{verbatim}

\begin{enumerate}
\def\labelenumi{(\alph{enumi})}
\tightlist
\item
\end{enumerate}

\begin{Shaded}
\begin{Highlighting}[]
\NormalTok{wheeze }\OtherTok{\textless{}{-}}\NormalTok{ wheeze }\SpecialCharTok{\%\textgreater{}\%}
  \FunctionTok{mutate}\NormalTok{(}\AttributeTok{smoke =} \FunctionTok{as.factor}\NormalTok{(smoke)) }

\CommentTok{\# \%\textgreater{}\%}
\CommentTok{\#   mutate(wheeze = as.factor(wheeze)) \%\textgreater{}\%}
\CommentTok{\#   mutate(kingston = as.factor(kingston))}


\NormalTok{wheeze\_binom }\OtherTok{\textless{}{-}} \FunctionTok{glm}\NormalTok{(wheeze }\SpecialCharTok{\textasciitilde{}}\NormalTok{ ., }\AttributeTok{data =}\NormalTok{ wheeze, }\AttributeTok{family =} \StringTok{"binomial"}\NormalTok{)}

\FunctionTok{summary}\NormalTok{(wheeze\_binom)}
\end{Highlighting}
\end{Shaded}

\begin{verbatim}
## 
## Call:
## glm(formula = wheeze ~ ., family = "binomial", data = wheeze)
## 
## Deviance Residuals: 
##     Min       1Q   Median       3Q      Max  
## -1.3820  -0.8890  -0.6402   1.2748   1.7231  
## 
## Coefficients: (1 not defined because of singularities)
##             Estimate Std. Error z value Pr(>|z|)
## (Intercept) -0.03729    0.99698  -0.037    0.970
## case         0.03012    0.07016   0.429    0.668
## t           -0.25530    0.30083  -0.849    0.396
## kingston     0.61103    0.64974   0.940    0.347
## age               NA         NA      NA       NA
## smoke1      -1.16160    0.72795  -1.596    0.111
## smoke2      -0.67572    0.99893  -0.676    0.499
## 
## (Dispersion parameter for binomial family taken to be 1)
## 
##     Null deviance: 62.687  on 49  degrees of freedom
## Residual deviance: 58.756  on 44  degrees of freedom
##   (14 observations deleted due to missingness)
## AIC: 70.756
## 
## Number of Fisher Scoring iterations: 4
\end{verbatim}

The coefficients for \texttt{smoke1} and \texttt{smoke2} both have a
p-value of greater than 0.05, thus smoking is not significant at the
95\% level.

\begin{enumerate}
\def\labelenumi{(\alph{enumi})}
\setcounter{enumi}{1}
\tightlist
\item
\end{enumerate}

\begin{Shaded}
\begin{Highlighting}[]
\FunctionTok{library}\NormalTok{(lme4)}
\end{Highlighting}
\end{Shaded}

\begin{verbatim}
## Warning: package 'lme4' was built under R version 4.1.2
\end{verbatim}

\begin{verbatim}
## Loading required package: Matrix
\end{verbatim}

\begin{verbatim}
## Warning: package 'Matrix' was built under R version 4.1.2
\end{verbatim}

\begin{Shaded}
\begin{Highlighting}[]
\NormalTok{wheeze\_binom\_random\_intercept }\OtherTok{\textless{}{-}} \FunctionTok{glmer}\NormalTok{(wheeze }\SpecialCharTok{\textasciitilde{}}\NormalTok{ t }\SpecialCharTok{+}\NormalTok{ kingston }\SpecialCharTok{+}\NormalTok{ smoke }\SpecialCharTok{+}\NormalTok{ (}\DecValTok{1}\SpecialCharTok{|}\NormalTok{case), }\AttributeTok{family =} \StringTok{\textquotesingle{}binomial\textquotesingle{}}\NormalTok{, }\AttributeTok{data =}\NormalTok{ wheeze)}
\FunctionTok{summary}\NormalTok{(wheeze\_binom\_random\_intercept)}\SpecialCharTok{$}\NormalTok{coefficients}
\end{Highlighting}
\end{Shaded}

\begin{verbatim}
##               Estimate Std. Error    z value  Pr(>|z|)
## (Intercept) -0.2351322  1.2115068 -0.1940825 0.8461113
## t           -0.2462958  0.3359447 -0.7331440 0.4634706
## kingston     0.8737005  1.0518742  0.8306131 0.4061922
## smoke1      -1.0753774  0.9085556 -1.1836121 0.2365667
## smoke2      -0.8899672  1.3649262 -0.6520258 0.5143845
\end{verbatim}

The standard errors and p-values get higher in the random intercept
model. Here the standard errors for \texttt{smoke1} and \texttt{smoke2}
are 0.91 and 1.36, respectively, compared to 0.73 and 1.0 from the GLM.
The p-values are 0.24 and 0.51 for the random intercept model, compared
to 0.11 and 0.50 from the GLM. Perhaps this is due to the lower number
of observations as the observations go from 64 (total observations) to 4
(grouped by individual child).

\hypertarget{appendix-all-code-for-this-report}{%
\section{Appendix: All code for this
report}\label{appendix-all-code-for-this-report}}

\begin{Shaded}
\begin{Highlighting}[]
\NormalTok{knitr}\SpecialCharTok{::}\NormalTok{opts\_chunk}\SpecialCharTok{$}\FunctionTok{set}\NormalTok{(}\AttributeTok{echo =} \ConstantTok{TRUE}\NormalTok{)}
\FunctionTok{library}\NormalTok{(dplyr)}
\CommentTok{\# Regression for cars data}
\NormalTok{lm.out }\OtherTok{\textless{}{-}} \FunctionTok{lm}\NormalTok{(dist }\SpecialCharTok{\textasciitilde{}}\NormalTok{ speed, }\AttributeTok{data =}\NormalTok{ cars)}

\CommentTok{\# create new predictor }
\NormalTok{newx }\OtherTok{=} \FunctionTok{seq}\NormalTok{(}\FunctionTok{min}\NormalTok{(cars}\SpecialCharTok{$}\NormalTok{speed), }\FunctionTok{max}\NormalTok{(cars}\SpecialCharTok{$}\NormalTok{speed), }\AttributeTok{by =} \FloatTok{0.1}\NormalTok{)}

\CommentTok{\# get confidence interval for each element in newx}
\NormalTok{conf\_interval }\OtherTok{\textless{}{-}} \FunctionTok{predict}\NormalTok{(lm.out, }
                         \AttributeTok{newdata=}\FunctionTok{data.frame}\NormalTok{(}\AttributeTok{speed=}\NormalTok{newx),}
                         \AttributeTok{interval =} \StringTok{"confidence"}\NormalTok{,}
                         \AttributeTok{level =} \FloatTok{0.95}\NormalTok{)}

\CommentTok{\# get prediction interval for each element in newx}
\NormalTok{pred\_interval }\OtherTok{\textless{}{-}} \FunctionTok{predict}\NormalTok{(lm.out,}
                         \AttributeTok{newdata =} \FunctionTok{data.frame}\NormalTok{(}\AttributeTok{speed=}\NormalTok{newx),}
                         \AttributeTok{interval =} \StringTok{"prediction"}\NormalTok{,}
                         \AttributeTok{level =} \FloatTok{0.95}\NormalTok{)}
\CommentTok{\# plot data points}
\FunctionTok{plot}\NormalTok{(cars}\SpecialCharTok{$}\NormalTok{speed, cars}\SpecialCharTok{$}\NormalTok{dist, }
     \AttributeTok{main =} \StringTok{"Regression"}\NormalTok{,}
     \AttributeTok{xlab =} \StringTok{"Speed"}\NormalTok{,}
     \AttributeTok{ylab =} \StringTok{"Distance"}\NormalTok{)}

\CommentTok{\# regression line}
\FunctionTok{abline}\NormalTok{(lm.out, }\AttributeTok{col =} \StringTok{\textquotesingle{}lightblue\textquotesingle{}}\NormalTok{)}

\CommentTok{\# add lower end of CI}
\FunctionTok{lines}\NormalTok{(newx, conf\_interval[,}\DecValTok{2}\NormalTok{], }\AttributeTok{col =} \StringTok{\textquotesingle{}blue\textquotesingle{}}\NormalTok{, }\AttributeTok{lty =} \DecValTok{2}\NormalTok{)}
\CommentTok{\# add upper end of CI}
\FunctionTok{lines}\NormalTok{(newx, conf\_interval[,}\DecValTok{3}\NormalTok{], }\AttributeTok{col =} \StringTok{\textquotesingle{}blue\textquotesingle{}}\NormalTok{, }\AttributeTok{lty =} \DecValTok{2}\NormalTok{)}
\CommentTok{\# add lower end of PI}
\FunctionTok{lines}\NormalTok{(newx, pred\_interval[,}\DecValTok{2}\NormalTok{], }\AttributeTok{col =} \StringTok{\textquotesingle{}red\textquotesingle{}}\NormalTok{, }\AttributeTok{lty =} \DecValTok{2}\NormalTok{)}
\CommentTok{\# add upper end of PI}
\FunctionTok{lines}\NormalTok{(newx, pred\_interval[,}\DecValTok{3}\NormalTok{], }\AttributeTok{col =} \StringTok{\textquotesingle{}red\textquotesingle{}}\NormalTok{, }\AttributeTok{lty =} \DecValTok{2}\NormalTok{)}
\CommentTok{\# library(matlib)}
\FunctionTok{set.seed}\NormalTok{(}\DecValTok{2022}\NormalTok{)}
\NormalTok{x }\OtherTok{=} \FunctionTok{rnorm}\NormalTok{(}\DecValTok{20}\NormalTok{, }\DecValTok{30}\NormalTok{, }\DecValTok{5}\NormalTok{)}
\NormalTok{y }\OtherTok{=} \FunctionTok{rnorm}\NormalTok{(}\DecValTok{20}\NormalTok{, }\DecValTok{40}\NormalTok{, }\DecValTok{5}\NormalTok{)}


\NormalTok{lm.out}\FloatTok{.3}\NormalTok{b }\OtherTok{=} \FunctionTok{lm}\NormalTok{(y }\SpecialCharTok{\textasciitilde{}}\NormalTok{ x)}
\NormalTok{xm }\OtherTok{=} \FunctionTok{matrix}\NormalTok{(}\FunctionTok{c}\NormalTok{(}\FunctionTok{rep}\NormalTok{(}\DecValTok{1}\NormalTok{, }\AttributeTok{times =} \DecValTok{20}\NormalTok{), x), }\AttributeTok{nrow =} \DecValTok{20}\NormalTok{)}
\NormalTok{ym }\OtherTok{=} \FunctionTok{as.matrix}\NormalTok{(y)}
\NormalTok{q  }\OtherTok{=} \FunctionTok{diag}\NormalTok{(}\AttributeTok{x =}\NormalTok{ x}\SpecialCharTok{\^{}}\DecValTok{2}\NormalTok{, }\AttributeTok{nrow =} \DecValTok{20}\NormalTok{, }\AttributeTok{ncol =} \DecValTok{20}\NormalTok{)}

\NormalTok{beta.hat }\OtherTok{=} \FunctionTok{solve}\NormalTok{(}\FunctionTok{t}\NormalTok{(xm)}\SpecialCharTok{\%*\%}\FunctionTok{solve}\NormalTok{(q)}\SpecialCharTok{\%*\%}\NormalTok{xm)}\SpecialCharTok{\%*\%}\FunctionTok{t}\NormalTok{(xm)}\SpecialCharTok{\%*\%}\FunctionTok{solve}\NormalTok{(q)}\SpecialCharTok{\%*\%}\NormalTok{ym}
\NormalTok{beta.hat}
\NormalTok{lm.out}\FloatTok{.3}\NormalTok{c }\OtherTok{=} \FunctionTok{lm}\NormalTok{(y }\SpecialCharTok{\textasciitilde{}}\NormalTok{ x, }\AttributeTok{weights =} \DecValTok{1}\SpecialCharTok{/}\NormalTok{(x}\SpecialCharTok{**}\DecValTok{2}\NormalTok{))}
\NormalTok{lm.out}\FloatTok{.3}\NormalTok{c}
\NormalTok{wheeze }\OtherTok{\textless{}{-}} \FunctionTok{read.csv}\NormalTok{(}\StringTok{\textquotesingle{}data/wheeze\_wmiss.csv\textquotesingle{}}\NormalTok{)}
\NormalTok{wheeze }\SpecialCharTok{\%\textgreater{}\%} \FunctionTok{head}\NormalTok{()}
\FunctionTok{summary}\NormalTok{(wheeze)}
\NormalTok{wheeze }\OtherTok{\textless{}{-}}\NormalTok{ wheeze }\SpecialCharTok{\%\textgreater{}\%}
  \FunctionTok{mutate}\NormalTok{(}\AttributeTok{smoke =} \FunctionTok{as.factor}\NormalTok{(smoke)) }

\CommentTok{\# \%\textgreater{}\%}
\CommentTok{\#   mutate(wheeze = as.factor(wheeze)) \%\textgreater{}\%}
\CommentTok{\#   mutate(kingston = as.factor(kingston))}


\NormalTok{wheeze\_binom }\OtherTok{\textless{}{-}} \FunctionTok{glm}\NormalTok{(wheeze }\SpecialCharTok{\textasciitilde{}}\NormalTok{ ., }\AttributeTok{data =}\NormalTok{ wheeze, }\AttributeTok{family =} \StringTok{"binomial"}\NormalTok{)}

\FunctionTok{summary}\NormalTok{(wheeze\_binom)}
\FunctionTok{library}\NormalTok{(lme4)}
\NormalTok{wheeze\_binom\_random\_intercept }\OtherTok{\textless{}{-}} \FunctionTok{glmer}\NormalTok{(wheeze }\SpecialCharTok{\textasciitilde{}}\NormalTok{ t }\SpecialCharTok{+}\NormalTok{ kingston }\SpecialCharTok{+}\NormalTok{ smoke }\SpecialCharTok{+}\NormalTok{ (}\DecValTok{1}\SpecialCharTok{|}\NormalTok{case), }\AttributeTok{family =} \StringTok{\textquotesingle{}binomial\textquotesingle{}}\NormalTok{, }\AttributeTok{data =}\NormalTok{ wheeze)}
\FunctionTok{summary}\NormalTok{(wheeze\_binom\_random\_intercept)}\SpecialCharTok{$}\NormalTok{coefficients}
\end{Highlighting}
\end{Shaded}


\end{document}
